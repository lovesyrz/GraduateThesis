\cleardoublepage{}
\begin{center}
    \bfseries \zihao{3} 摘~要
\end{center}


本论文深入研究了一类分形函数水平集的基本性质,旨在拓展和深化于汉在《Weak tangent and level sets of Takagi functions》中关于 Takagi 函数水平集的研究成果。通过构建数学模型、运用分形几何学中的维数理论以及拓扑学的相关理论和方法,我们对一类与Takagi 函数类似的分形函数的水平集性质进行了系统分析。

在研究过程中,我们首先详细探讨了分形函数的自相似性、不可微性等基本特性,并分析了这些特性对水平集性质的影响。接着,我们重点研究了一类以可分割折线函数拓展出的类Takagi函数的水平集的维数特征,成功证明了在满足特定条件下,该类分形函数存在一个水平集,其 Hausdorff 维数具有明确的下界估计。此外,我们还借助相关数学理论,推导出了该类分形函数图像的 Assouad 维数的下界,进一步完善了对该类函数整体性质的认识。

虽然本研究取得了一些理论成果,但仍存在一些不足之处。一方面,水平集的维数的精确计算、在更广泛的参数取值范围内的水平集性质以及拓扑结构的深入分析等方面仍有待进一步研究,另一方面,如何将本文内容与实际应用结合仍需继续研究。

\noindent{\textbf{关键词:}Takagi函数, 广义Takagi函数, Hausdorff维数, Assouad维数, 水平集}
\cleardoublepage{}
\begin{center}
    \bfseries \zihao{3} Abstract
\end{center}

This thesis delves into the fundamental properties of the level sets of a class of fractal functions, aiming to expand and deepen the research findings on the level sets of the Takagi function presented by Han in “Weak Tangent and Level Set of Takagi Function.” By constructing mathematical models and employing the dimension theory in fractal geometry as well as relevant theories and methods from topology, we have systematically analyzed the properties of the level sets of a class of fractal functions similar to the Takagi function.

During the research process, we first thoroughly explored the basic characteristics of fractal functions, such as self-similarity and non-differentiability, and analyzed the impact of these characteristics on the properties of level sets. Subsequently, we focused on the dimension features of level sets of a class of Takagi-like functions expanded from piecewise linear functions, successfully proving that under specific conditions, a level set exists for this class of fractal functions, with a lower bound estimate for its Hausdorff dimension. Moreover, by leveraging relevant mathematical theories, we derived the lower bound of the Assouad dimension of the graph of this class of fractal functions, further enhancing our understanding of the overall properties of these functions.

While this study has achieved some theoretical results, there are still some shortcomings. On the one hand, the precise calculation of the dimension of level sets, the properties of level sets within a broader range of parameter values, and in-depth analysis of topological structures all require further investigation. On the other hand, how to combine the content of this paper with practical applications still needs to be further studied.

\noindent{\textbf{Keywords:}Takagi function, Generalized Takagi function, Hausdorff dimension, Assouad dimension, Level set}