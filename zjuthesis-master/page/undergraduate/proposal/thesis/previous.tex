{
    \setlength{\parindent}{0em}
    \par {\zihao{4}\bfseries 一、题目:\Title}
    \par {\zihao{4}\bfseries 二、指导教师对文献综述、开题报告、外文翻译的具体要求:}

    根据阅读的国内外文献撰写文献综述,要求根据主题展开,文献综述内容要切题,包括\\
          (1)简述与分形函数水平集相关的近几年论文的研究内容和相关进展,掌握当前该领域的研究前沿,分析你毕业论文研究的内容与这些论文的差异和相关。\\
          (2)分析以上论文所采用的研究方法,简述这些方法的优劣和你的思考。\\
          (3)简述各参考文献的创新性、存在的问题或未能解决的问题。\\
          (4)要求翻译其中的一篇外文文献,结构完整,语句通顺。
    推荐的参考文献:\newline
          [1] Y. Han, Weak tangent and level sets of Takagi functions, Monatsh. Math., 192 (2020), 249—264. (读懂这篇论文,翻译这篇论文的前5节).\newline
          [2] P. Allaart, K. Kawamura: The Takagi function: a survey. Real Anal. Exch. 37 (2011), 1–54. (粗读这篇论文)\newline
          [3] J.M. Fraser, Assouad dimension and fractal geometry, Cambridge Tracts in Mathematics, vol. 222, Cambridge University Press, Cambridge, 2021.\newline
          [4] K. Falconer, Fractal geometry: Mathematical foundations and applications, 3rd ed., John Wiley \& Sons, Ltd., Chichester, 2014.


    开题报告要求:\\
          (1)分析近几年对于分形函数水平集取得的进展,给出具体要研究的问题的意义(包括理论意义和实际意义,并分析课题与本专业的关系)\\
          (2) 根据文献综述分析课题的研究背景(即要解决什么问题?本文所讨论问题的角度与已有参考文献中所涉及的问题的差异,课题的主要创新点是什么?)\\
          (3) 选题的可行性分析(一般从研究技术的可行性等方面去说明)。\\
          (4) 主要研究内容(这部分要展开写,主要包括理论研究内容、实证分析内容等)。\\
          (5) 根据研究的内容写出具体的实施计划。\\
          (6) 明确论文最后预期结果。


}

\mbox{} \vfill

\signature{指导教师(签名)}
% comment the line above and uncomment the line below if you want to set a signature with a specific date.
% \signaturewithdate{指导教师(签名)}{1897}{5}{21}
