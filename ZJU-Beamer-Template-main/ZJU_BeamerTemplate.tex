%{\tiny }%# -*- coding:utf-8 -*-
\documentclass[10pt,aspectratio=169,mathserif]{beamer}
%设置为 Beamer 文档类型,设置字体为 10pt,长宽比为16:9,数学字体为 serif 风格

%%%%-----导入宏包-----%%%%
\usepackage{zju}			%导入 zju 模板宏包
\usepackage{ctex}			%导入 ctex 宏包,添加中文支持
\usepackage{amsmath,amsfonts,amssymb,bm}   %导入数学公式所需宏包
\usepackage{color}			 %字体颜色支持
\usepackage{graphicx,hyperref,url}
\usepackage{metalogo}	% 非必须
%% 上文引用的包可按实际情况自行增删
%%%%%%%%%%%%%%%%%%
\usepackage{fontspec}
\usepackage{xeCJK}
% \setCJKmainfont{Source Han Sans SC}



\beamertemplateballitem		%设置 Beamer 主题

%%%%------------------------%%%%%
\catcode`\。=\active         %或者=13
\newcommand{。}{.}
%将正文中的“。”号转换为“.”。中文标点国家规范建议科技文献中的句号用圆点替代
%%%%%%%%%%%%%%%%%%%%%

%%%%----首页信息设置----%%%%
\title[ 分形函数水平集的基本性质]{分形函数水平集的基本性质}
\subtitle{——从Takagi函数到广义的Takagi函数}
%%%%----标题设置


\author[Kangjie Lou]{
  楼康杰 \\\medskip
  指导老师:阮火军
  }
%%%%----个人信息设置

\institute[IOPP]{
  数学科学学院 \\
  浙江大学}
%%%%----机构信息

\date[May 16 2025]{
  2025年5月16日}
%%%%----日期信息

\begin{document}

	\begin{frame}
		\titlepage
	\end{frame}				%生成标题页

	\section{提纲}
	\begin{frame}
		\frametitle{提纲}
		\tableofcontents
	\end{frame}				%生成提纲页


	\section{Takagi函数及其水平集的性质}
	\begin{frame}[plain]
		\frametitle{Takagi函数}
		\begin{theorem}[Takagi函数]
			定义以下函数$T_{a,b}(x)$为Takagi函数,
			$$
				T_{a,b}(x)=\sum_{n=0}^\infty a^nT(b^nx),
			$$
			其中$0<a<1,b>1$,$T(x):\mathbb{R}\rightarrow\mathbb{R}$周期为$1$,在区间$[0,1]$上定义如下:
			$$
				T(x)=\begin{cases}
					x~~~~~~,x\in[0,\frac{1}{2}],\\
					1-x~,x\in[\frac{1}{2},1].
				\end{cases}
			$$
		\end{theorem}

		\begin{theorem}[水平集]
			函数$f:[0,1]\rightarrow\mathbb{R}$在$f(x)=y$时的水平集定义为
			$$
				L(y)=\{x\in[0,1]:f(x)=y\}\times\{y\}.
			$$
		\end{theorem}
	\end{frame}


\end{document}