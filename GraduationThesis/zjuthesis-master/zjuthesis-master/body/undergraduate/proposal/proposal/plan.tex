\section{研究计划进度安排及预期目标}

\subsection{进度安排}
\label{sec:schedule}

本研究计划分为三个主要阶段,具体安排如下:

\begin{enumerate}
      \item \textbf{第一阶段(第1 - 2周):文献综述与理论基础}
      \begin{itemize}
            \item 查阅相关文献,了解分形函数水平集的研究现状和发展趋势。
            \item 深入研究分形函数水平集的理论知识,推导其基本性质。
      \end{itemize}
      \item \textbf{第二阶段(第3 - 4周):数值模拟与实证分析}
      \begin{itemize}
            \item 进行数值模拟和计算机实验,验证理论分析的正确性。
            \item 撰写实证分析部分的论文内容。
      \end{itemize}
      \item \textbf{第三阶段(第5-6周):结果整理与论文撰写}
      \begin{itemize}
            \item 整理研究结果,撰写毕业论文。
            \item 准备论文答辩,进行最终的修订和调整。
      \end{itemize}
\end{enumerate}

\subsection{预期目标}
\label{sec:expected_goals}

本研究的预期目标如下:

\begin{enumerate}
      \item \textbf{理论成果}
      \begin{itemize}
            \item 系统阐述分形函数水平集的维数性质,包括豪斯多夫维数、盒维数和Assouad维数。
            \item 尝试提出新的理论观点和方法,丰富分形几何的理论体系。
      \end{itemize}
      \item \textbf{实证成果}
      \begin{itemize}
            \item 通过数值模拟验证理论分析的正确性,展示分形函数水平集在不同参数下的变化规律。
            \item 提供详细的数值模拟结果和数据分析报告。
      \end{itemize}
      \item \textbf{应用成果}
      \begin{itemize}
            \item 探讨分形函数水平集在图像处理、信号分析等领域的应用前景。
      \end{itemize}
\end{enumerate}

通过上述研究计划和预期目标,本项目旨在系统地研究分形函数水平集的维数性质,为分形几何的理论体系增添新的内容,并为相关领域的应用提供新的视角和方法。