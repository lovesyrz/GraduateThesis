\cleardoublepage

\section{绪论}

\subsection{背景}

\subsubsection{分形几何的起源与发展}

分形几何学是一门以不规则几何形态为研究对象的新兴学科,其起源可以追溯到19世纪末和20世纪初的数学研究。当时,一些数学家在研究中发现了许多具有自相似性和复杂结构的几何对象,这些对象无法用传统的欧几里得几何来描述。例如,康托尔集、魏尔斯特拉斯函数和科赫曲线等,这些对象虽然在当时被认为是“病态”的数学构造,但它们却蕴含着分形几何的基本思想。直到20世纪70年代,曼德勃罗(Benoît B. Mandelbrot)在他的著作《分形:形状、机遇和维数》中首次提出了“分形”这一概念,并系统地发展了分形几何学的理论框架,才使得分形几何真正成为了一门独立的学科。分形几何学的出现,不仅为数学领域带来了新的研究视角和方法,也为物理学、生物学、地理学、经济学等众多学科中复杂不规则现象的研究提供了有力的工具和理论支持。

\subsubsection{分形函数的定义与重要性}

分形函数是分形几何学中的一个重要研究对象,它是一类具有分形特性的函数。这类函数通常具有自相似性、不可微性以及复杂的局部和全局结构。经典的分形函数如魏尔斯特拉斯函数、Takagi函数等,它们在数学上具有独特的性质,例如魏尔斯特拉斯函数是一个处处连续但处处不可导的函数,这种性质在传统的数学分析中是难以想象的。分形函数的研究不仅有助于我们深入理解数学分析中的一些基本概念和定理,如连续性、可微性等,还为研究自然界中的复杂现象提供了模型。例如,在地理学中,海岸线的形状可以用分形函数来描述;在物理学中,某些物理量的分布也表现出分形特性。因此,分形函数的研究具有重要的理论意义和实际应用价值。

\subsubsection{分形函数水平集的研究现状}

分形函数的水平集是指函数取相同值的所有点的集合。水平集的研究在分形几何学中具有重要的地位,它可以帮助我们更好地理解分形函数的结构和性质。近年来,随着分形几何学的不断发展,分形函数水平集的研究也取得了许多重要成果。例如,对于经典的Takagi函数,于汉在其文章《Weak Tangent and Level Set of Takagi Function》中,对该函数的水平集进行了深入研究,揭示了其水平集在某些特定情况下的维数性质以及弱切线的存在性等重要结论。这些研究成果不仅丰富了分形函数水平集理论的研究内容,也为后续相关研究提供了宝贵的思路和方法借鉴。然而,目前对于分形函数水平集的研究仍然存在许多未解决的问题,例如对于更广泛的分形函数类,其水平集的精确维数计算、水平集的拓扑结构以及水平集之间的相互关系等,这些问题都需要进一步深入研究。

\subsubsection{本研究的动机与目的}

基于上述研究背景和现状,本研究旨在进一步拓展和深化分形函数水平集的理论研究。具体而言,我们将研究对象聚焦于一类与Takagi函数类似的分形函数,尝试将于汉关于Takagi函数水平集的研究结论进行合理拓展和推广。通过深入分析这类分形函数的水平集性质,期望能够揭示其在不同参数条件下的维数特征、拓扑结构以及与其他数学对象的关系等,从而为分形函数水平集理论的发展提供新的见解和方法。此外,本研究还希望通过理论研究与实际应用相结合的方式,探索分形函数水平集在图像处理、信号分析等领域的潜在应用价值,为相关领域的研究和实践提供一定的理论支持和参考。

\subsection{研究方法与技术路线}

\subsubsection{研究方法}

在本研究中,我们将采用多种数学分析方法和工具来研究分形函数水平集的性质。首先,我们将运用经典的数学分析理论,如极限、导数、积分等概念和方法,对分形函数的基本性质进行深入分析。其次,我们将借助分形几何学中的维数理论,如盒维数、Hausdorff维数和Assouad维数等,来研究水平集的维数特征。此外,我们还将利用拓扑学中的相关理论和方法,探讨水平集的拓扑结构。在研究过程中,我们将结合具体的数学模型和实例,通过构造、证明和计算等手段,逐步推导出相关的结论。

\subsubsection{技术路线}

本研究的技术路线主要包括以下几个步骤:

\begin{enumerate}
      \item 文献综述与理论基础研究:首先,对已有的分形几何学和分形函数水平集的研究文献进行全面的综述和分析,总结前人的研究成果和研究方法,为本研究奠定坚实的理论基础。
      \item 数学模型的构建与分析:基于研究对象的特点,构建相应的数学模型,并运用数学分析方法对其基本性质进行分析。特别地,对于所研究的分形函数,将详细分析其自相似性、不可微性等特性,并探讨这些特性对水平集性质的影响。
      \item 水平集性质的深入研究:在数学模型的基础上,进一步研究分形函数的水平集性质。具体包括水平集的维数计算、拓扑结构分析以及水平集之间的相互关系等。通过构造具体的例子和进行严格的数学证明,得出相关的结论。
      \item 理论结果的应用与拓展:将所得到的理论结果与实际应用领域相结合,探讨其在图像处理、信号分析等领域的潜在应用价值。通过具体的案例分析,验证理论结果的有效性和实用性,并进一步拓展研究思路和方法。
      \item 总结与展望:对本研究的主要成果进行总结,并指出研究中存在的不足之处和未来的研究方向。
\end{enumerate}


\subsection{研究意义与预期成果}

\subsection{研究意义}

本研究的开展具有重要的理论意义和实际应用价值。从理论角度来看,通过对一类分形函数水平集性质的深入研究,将进一步丰富和完善分形函数水平集理论体系,为分形几何学的发展提供新的理论支持和研究思路。此外,本研究还将有助于深化我们对分形函数结构和性质的理解,推动相关数学分支的交叉融合与发展。从实际应用角度来看,分形函数水平集的研究成果有望在图像处理、信号分析等领域得到应用。例如,在图像处理中,利用分形函数水平集的性质可以实现图像的特征提取和分析;在信号分析中,可以用于复杂信号的建模和特征识别等。因此,本研究不仅具有重要的学术价值,还具有广阔的应用前景。

\subsubsection{预期成果}

本研究预期将取得以下几方面的成果:

\begin{enumerate}
      \item 理论成果:系统地研究一类分形函数水平集的性质,包括水平集的维数特征、拓扑结构以及与其他数学对象的关系等,并得出一系列具有创新性的理论结论。这些结论将为分形函数水平集理论的发展提供新的视角和方法。
      \item 方法成果:在研究过程中,总结和提炼出一套有效的研究方法和技巧,为后续相关研究提供参考和借鉴。特别是在分形函数水平集的维数计算和拓扑结构分析方面,形成一套较为完善的研究方法体系。
      \item 应用成果:探索分形函数水平集在图像处理、信号分析等领域的潜在应用价值,并通过具体的案例分析,验证理论结果的有效性和实用性。为相关领域的实际应用提供一定的理论支持和方法指导。
      \item 学术成果:撰写并发表相关的学术论文,参加学术会议进行学术交流,将研究成果推向学术界,促进学术交流与合作。
\end{enumerate}