\cleardoublepage{}
\begin{center}
    \bfseries \zihao{3} 摘~要
\end{center}

本论文深入研究了一类分形函数水平集的基本性质,旨在拓展和深化于汉在《Weak Tangent and Level Set of Takagi Function》中关于 Takagi 函数水平集的研究成果。通过构建数学模型、运用分形几何学中的维数理论以及拓扑学的相关理论和方法,我们对一类与 Takagi 函数类似的分形函数的水平集性质进行了系统分析。
在研究过程中,我们首先详细探讨了分形函数的自相似性、不可微性等基本特性,并分析了这些特性对水平集性质的影响。接着,我们重点研究了水平集的维数特征,成功证明了在满足特定条件下,该类分形函数存在一个水平集,其 Hausdorff 维数具有明确的下界估计。此外,我们还借助引理和相关数学工具,推导出了该类分形函数图像的 Assouad 维数的下界,进一步完善了对该类函数整体性质的认识。
除了理论研究,我们还积极探索了分形函数水平集在图像处理、信号分析等领域的潜在应用价值。通过具体的案例分析,验证了理论结果的有效性和实用性,展示了分形函数水平集理论在实际应用中的广阔前景。
尽管本研究取得了一系列重要的研究成果,但仍存在一些不足之处。例如,水平集的精确维数计算、更广泛的参数取值范围内的水平集性质以及拓扑结构的深入分析等方面仍有待进一步研究。未来的研究将致力于解决这些问题,并进一步拓展分形函数水平集理论的应用领域。
总之,本论文不仅丰富了分形函数水平集的理论研究内容,还为相关领域的实际应用提供了新的理论支持和方法指导,具有重要的理论意义和实际应用价值。

\cleardoublepage{}
\begin{center}
    \bfseries \zihao{3} Abstract
\end{center}

This thesis delves into the fundamental properties of the level sets of a class of fractal functions, aiming to expand and deepen the research findings on the level sets of the Takagi function presented by Han in "Weak Tangent and Level Set of Takagi Function." By constructing mathematical models and employing the dimension theory in fractal geometry as well as relevant theories and methods from topology, we have systematically analyzed the properties of the level sets of a class of fractal functions similar to the Takagi function.

During the research process, we first thoroughly explored the basic characteristics of fractal functions, such as self-similarity and non-differentiability, and analyzed the impact of these characteristics on the properties of level sets. Subsequently, we focused on the dimension features of level sets, successfully proving that under specific conditions, a level set exists for this class of fractal functions, with a lower bound estimate for its Hausdorff dimension. Moreover, by leveraging lemmas and relevant mathematical tools, we derived the lower bound of the Assouad dimension of the graph of this class of fractal functions, further enhancing our understanding of the overall properties of these functions.

In addition to theoretical research, we actively explored the potential applications of the level sets of fractal functions in the fields of image processing and signal analysis. Through case studies, we verified the effectiveness and practicality of the theoretical results, demonstrating the broad application prospects of the theory of level sets of fractal functions.

Despite the significant research achievements obtained in this thesis, there are still some shortcomings. For example, the precise calculation of the dimension of level sets, the properties of level sets within a broader range of parameter values, and in-depth analysis of topological structures all require further investigation. Future research will be dedicated to addressing these issues and further expanding the application fields of the theory of level sets of fractal functions.

In summary, this thesis not only enriches the theoretical research on the level sets of fractal functions but also provides new theoretical support and methodological guidance for practical applications in related fields, holding significant theoretical and practical value.